\documentclass{beamer}

\usepackage[style=numeric, natbib]{biblatex}
\usepackage{caption}
\usepackage{color}
\usepackage{amsmath}
\usepackage{array}
\usepackage{booktabs}
\usepackage{caption}
\usepackage{fancyvrb}
\usepackage{graphicx}
\usepackage{longtable}
\usepackage{newtxmath}
\usepackage{tabularx}
\usepackage{zi4}
\usepackage{graphicx}
\usepackage{colortbl}
\usepackage{microtype}
\usepackage{balance}
\usepackage{mathtools}
\usepackage{float}
\usepackage{listings,lstlangcoq}
\usepackage{xcolor}
\usepackage{centernot}
\usepackage[all]{xy}

\lstset{language=sh,
  basicstyle=\footnotesize\ttfamily,
  keywordstyle=\footnotesize\color{blue}\ttfamily,
  moredelim=**[is][\btHL]{|}{|},
  escapeinside={<@}{@>}
}

\usetheme{Antibes}
\usecolortheme{beaver}

\addbibresource{references.bib}

\title{CFG Patterns: A new tool to formally verify optimizations in Vellvm}

\author[Léon Frenot]{Leon Frenot\\ Supervised by Yannick Zakowski \& Gabriel Radanne}

\date[July 12th, 2024]{February 5th, 2024 - July 5th, 2024}

% \setbeamertemplate{footline}[frame number]
\setbeamertemplate{navigation symbols}{}

\makeatletter
\setbeamertemplate{footline}{
  \leavevmode%
  \hbox{%
    \begin{beamercolorbox}[wd=.333333\paperwidth,ht=2.25ex,dp=1ex,center]{author in head/foot}%
      \usebeamerfont{author in head/foot}\insertshortauthor
    \end{beamercolorbox}%
    \begin{beamercolorbox}[wd=.333333\paperwidth,ht=2.25ex,dp=1ex,center]{title in head/foot}%
      \usebeamerfont{title in head/foot}\insertshortdate
    \end{beamercolorbox}%
    \begin{beamercolorbox}[wd=.333333\paperwidth,ht=2.25ex,dp=1ex,right]{date in head/foot}%
      \insertframenumber{} / \inserttotalframenumber\hspace*{2ex}
    \end{beamercolorbox}%
  }%
  \vskip0pt%
}
\makeatother

\lstdefinestyle{customcoq}{
  columns=flexible,
  mathescape=true,
  belowcaptionskip=1\baselineskip,
  breaklines=true,
  xleftmargin=\parindent,
  language=Coq,
  morekeywords={Variant, fun, Arguments, Type, cofix},
  % morekeywords={SOCKAPI,ITREE,data_at,data_at_},
  emph={%
      SOCKAPI,ITree,data_at,data_at_
    },
  emphstyle={\bfseries\color{green!40!red!80}},
  showstringspaces=false,
  basicstyle=\small\ttfamily,
  keywordstyle=\bfseries\color{green!20!black},
  commentstyle=\itshape\color{red!40!black},
  identifierstyle=\color{violet!50!black},
  stringstyle=\color{orange},
  escapeinside={<@}{@>}
}
\newcommand{\inlinecoq}[1]{\mbox{\lstinline[style=customcoq,columns=fixed,basewidth=.48em]{#1}}}
\newcommand{\ilc}[1]{\inlinecoq{#1}}

\newcommand{\leon}[1]{\textcolor{blue}{#1}}
\newcommand{\gr}[1]{\textcolor{Orange}{#1}}
\newcommand{\yz}[1]{\textcolor{ForestGreen}{#1}}
\newcommand{\yzt}[1]{\textcolor{ForestGreen!50}{#1}}
\newcommand{\cut}[1]{\textcolor{Gray!40}{#1}}
\newcommand{\ocfg}{OCFG\xspace}
\newcommand{\pat}{\texttt{Pat}\xspace}

\begin{document}

\frame{\titlepage}

\section*{Introduction}

\begin{frame}
  \begin{block}{Context}
    \begin{itemize}
      \item M2 research internship
      \item 5 Months in LIP Computer Science laboratory at ENS Lyon.
      \item Compilation and Analysis, Software and Hardware Team.
    \end{itemize}
  \end{block}
  \begin{block}{Keywords}
    \begin{itemize}
      \item Verified Compilation
      \item Verified Code transformations
      \item DSL of patterns
    \end{itemize}
  \end{block}
\end{frame}

\section{Introduction}

\begin{frame}{Table of Contents}
  \tableofcontents
\end{frame}

\subsection*{Motivation}

\begin{frame}{Motivation}
  \begin{itemize}
    \item GCC\@: 15 million lines of code
    \item CompCert (2009)
    \item Vellvm (2021)
  \end{itemize}
\end{frame}

\subsection*{Goal of the internship}

\begin{frame}
  \begin{block}{Goal of the internship}
    Formally verify optimizations in Vellvm
  \end{block}
  \begin{itemize}
    \item Identify a subgraph with specific structural properties.
    \item Link the transformed subgraph back with the context.
  \end{itemize}
  % \leon{DSL Pat, matcher, implementation, preserve semantics, generic substitution theorem}
\end{frame}

\begin{frame}{Contributions}
  \begin{itemize}
    \item Designed and implemented a DSL of patterns over graphs, Pat.
    \item Equipped it with a matcher that computes the set of decompositions that match a given pattern in a given graph.
    \item Proved a generic substitution theorem.
    \item Leveraged it and the specifications given by Pat to establish a proof for an optimization.
  \end{itemize}
\end{frame}

\section{Background}

\begin{frame}{Table of Contents}
  \tableofcontents[currentsection]
\end{frame}

\subsection*{LLVM IR}

\begin{frame}{LLVM IR}
  \begin{itemize}
    \item Intermediate representation for multiple frontends and backends.
    \item RISC-like instruction set with high-level informations.
    \item Features:\begin{itemize}
            \item SSA form guarantee
            \item named labels and registers
            \item integer-pointer casts
            \item analysis and transformation passes
          \end{itemize}
    \item Vellvm: formally define LLVM IR's semantics and build verified components.
  \end{itemize}
\end{frame}

\subsection*{ITrees}

\begin{frame}[fragile]{Itrees}
  \begin{figure}
    \begin{lstlisting}[style=customcoq,basicstyle=\small\ttfamily]
CoInductive itree (E : Type -> Type) (R : Type) : Type :=
  | Ret (r : R)
  | Tau (t : itree E R)
  | Vis {A : Type} (e : E A) (k : A -> itree E R).
  \end{lstlisting}
    \caption{The \ilc{itree} datatype}
    \label{fig:itree}
  \end{figure}
  \leon{coinductive $\rightarrow$ potentialy infinite $\rightarrow$ recursive and effectful programs including divergent computations, example figure (infinite computation, echo), effects/event handler, bisimilarity, denotational semantics}
\end{frame}

\subsection*{Vellvm}

\begin{frame}{Vellvm's semantics}
  \begin{itemize}
    \item Verified IR: a realistic subset of LLVM IR.
    \item Syntax captured by ITrees, effects captured by Vis.
    \item Supports non-trivial features: pointers, $\phi$-nodes, undef \ldots
  \end{itemize}
  \leon{stack of interp (+ figure), level 0}
\end{frame}

\section{The pattern language}

\begin{frame}{Table of Contents}
  \tableofcontents[currentsection]
\end{frame}

\subsection*{The Pattern datatype}

\begin{frame}[fragile]{Pat}
  \begin{figure}
    \begin{lstlisting}[style=customcoq,basicstyle=\small\ttfamily]
Inductive Pattern : Type -> Type :=
  |Graph: Pattern ocfg
  |When: forall  {S}, Pattern S -> (S -> bool) -> Pattern S
  |Map: forall  {S} {T}, Pattern S -> (S -> T) -> Pattern T
  |Focus: forall  {S}, Pattern S -> Pattern (ocfg * S)
  |Block: forall  {S}, Pattern S -> Pattern (block_id * blk * S)
  |Head: forall  {S}, Pattern S -> Pattern (block_id * blk * S)
  |Branch: forall  {S}, Pattern S -> Pattern (block_id * blk * S)
  \end{lstlisting}
    \caption{The \ilc{Pattern} datatype}
    \label{fig:pat}
  \end{figure}
  \leon{dependent type, explication for each pattern}
\end{frame}

\begin{frame}[fragile]{A pattern for Block Fusion}
  \begin{figure}
    \scalebox{0.75}{\xymatrix{
    &&&&\\
    &*+[F]\txt{\ilc{A}\\\ilc{Block}: any block}\ar[dd]&&\;\ar@{-->}`u[ul] `[ll] [ll]&\\
    *+[F.]\txt{\ilc{When _ BlockFusion_f}\\ asserts that \ilc{B} is\\the only successor of \ilc{A}}\ar@{..>} [r]&&&\txt{Graph}\\
    &*+[F]\txt{\ilc{B}\\\ilc{Head}: no predecessors\\once \ilc{A} is removed}\ar@{-->}`d[dr] `[rr] [rr]&&&\\
    &&&
    \save "2,3"."4,5"*[F--]\frm{} \restore
    }}
  \end{figure}
  \begin{lstlisting}[style=customcoq,basicstyle=\small\ttfamily]
    pfusion := When (Block (Head Graph)) BlockFusion_f
    \end{lstlisting}
  % \leon{Figure, explanation}
\end{frame}

\subsection*{A Matcher function}

\begin{frame}[fragile]{MatchAll}
  \begin{figure}
    \begin{lstlisting}[style=customcoq,basicstyle=\small\ttfamily]
Fixpoint MatchAll {S} (P: Pattern S) (g: ocfg) : list S :=
  match P with
    | Graph => [g]
    | When p f => filter (fun x => f x = true) (MatchAll p g) 
    | Map p f => map f (MatchAll p g)
    | Focus p => flat_map_r (MatchAll p) (focus g)
    | Block p => flat_map_r (MatchAll p) (blocks g)
    | Head p => flat_map_r (MatchAll p) (heads g)
    | Branch p => flat_map_r (MatchAll p) (branches g)
  end.
  \end{lstlisting}
    \caption{The \ilc{MatchAll} function}
    \label{fig:matchall}
  \end{figure}
  \leon{Figure, explication for the patterns, detail one of them}
\end{frame}

\begin{frame}[fragile]{An example: focus}
  \begin{figure}
    \begin{lstlisting}[style=customcoq,basicstyle=\small\ttfamily]
Definition focus_aux id b acc : list ocfg :=
  acc ++ (map (insert id b) acc).

Definition focus (G: ocfg) :=
  map (fun G' => (G', G ∖ G')) (map_fold focus_aux [∅] G).
  \end{lstlisting}
    \caption{An implementation of focus}
  \end{figure}
\end{frame}

\subsection*{Specification}

\begin{frame}[fragile]{Specification}
  \begin{figure}
    \begin{lstlisting}[style=customcoq,basicstyle=\small\ttfamily]
forall (G:ocfg) (X:S), X ∈ MatchAll pat G IFF R_pat G X

forall (P: Pattern S) f X G, X ∈ (MatchAll (When P f) G) IFF
  f X = true \/ X ∈ (MatchAll P G).
  \end{lstlisting}
    \caption{The correctness statements for \ilc{Graph} and \ilc{When}}
  \end{figure}
\end{frame}

\begin{frame}[fragile]{Specification of Graph}
  \begin{figure}
    \begin{lstlisting}[style=customcoq,basicstyle=\small\ttfamily]
(* The semantics for heads_aux *)
Record heads_aux_sem (G0 G G': ocfg) id b := {
  EQ: G' = delete id G0;
  IN: G !! id = Some b;
  PRED: predecessors id G0 = ∅
}.

(* The semantics for heads and Head *)
Definition heads_sem (G G':ocfg) (id:block_id) b := heads_aux_sem G G G' id b.
    \end{lstlisting}
    \caption{The specification of \ilc{Head}}
    \label{fig:sem_head_def}
  \end{figure}
\end{frame}

\begin{frame}[fragile]
  \begin{figure}
    \begin{lstlisting}[style=customcoq,basicstyle=\small\ttfamily]
(* heads_aux follows its semantics *)
Lemma heads_aux_correct:
  forall G G' G0 id b,
  (id, b, G') ∈ (map_fold (heads_aux G0) [] G) IFF heads_aux_sem G0 G G' id b.

(* heads follows its semantics *)
Lemma heads_correct:
  forall G G' id b,
  (id, b, G') ∈ (heads G) IFF heads_sem G G' id b.

(* MatchAll Head follows the given semantics *)
Theorem Pattern_Head_correct {S}:
  forall (G: ocfg) (P: Pattern S) id b X,
  (id, b, X) ∈ (MatchAll (Head P) G) IFF
  exists G', heads_sem G G' id b /\ X ∈ (MatchAll P G').
      \end{lstlisting}
    \caption{The correctness statement of \ilc{MatchAll Head}}
    \label{fig:head_cor}
  \end{figure}
\end{frame}

\section{Formal verification of an optimization pass}

\begin{frame}{Table of Contents}
  \tableofcontents[currentsection]
\end{frame}

\subsection*{Substitution under context}

\begin{frame}{Substitution under context}
  \begin{figure}
    \xymatrix{
    &&&&&&\\
    G_2\approx G_2'&\implies&*+[F]{G_2}\ar@{-->}`d[dr] `[r] [r]&*+[F]{G_1}\ar@{-->}`u[ul] `[l] [l]&\approx&*+[F]{G_2'}\ar@{-->}`d[dr] `[r] [r]&*+[F]{G_1}\ar@{-->}`u[ul] `[l] [l]\\
    &&&&&&
    }
    \caption{Ideal substitution}
  \end{figure}
\end{frame}

\begin{frame}
  \begin{figure}
    \scalebox{0.6}{\xymatrix{
    *+[F]{A}\ar[dd]&&&&*+[F]{A}\ar[dd] &&&&\\
    &\approx& *+[F]{AB}&\centernot\implies&&*+[F]{C}\ar[dl]&\approx&*+[F]{AB}&*+[F]{C}\ar@{-->}[dl]\\
    *+[F]{B}&&&& *+[F]{B}&&&*+[F--]{B}&\\
    *+[F]{A}\ar[dd]&&&&*+[F]{A}\ar[dd] &&&*+[F--]{A}&\\
    &\approx& *+[F]{AB}&\centernot\implies&&*+[F]{C}\ar[ul]&\approx&*+[F]{AB}&*+[F]{C}\ar@{-->}[ul]\\
    *+[F]{B}&&&& *+[F]{B}&&&&\\
    *+[F]{A}\ar[dd]&&&&*+[F]{A}\ar[dd] &&&&\\
    &\approx& *+[F]{AB}&\centernot\implies&&*+[F]{\phi(B,x);;C}&\approx&*+[F]{AB}\ar[r]&*+[F]{\phi(B,x);;C}\\
    *+[F]{B}&&&& *+[F]{B}\ar[ur]&&&&
    }}
    \caption{Trying to apply the ideal substitution to Block Fusion}
  \end{figure}
\end{frame}

\begin{frame}[fragile]
  \begin{figure}
    \begin{lstlisting}[style=customcoq,basicstyle=\footnotesize\ttfamily]
Theorem denote_ocfg_equiv (g1 g2 g2' : ocfg)
  (σ : block_id -> block_id) (nTO: gset block_id) :

  (* Well formedness properties *)
  inputs g2' ∖ inputs g2 ⊆ nTO -> 
  nTO ⊆ inputs g2 ∪ inputs g2' -> 
  nTO ## outputs g1 ->
  g1 ## g2 -> 
  ocfg_term_rename σ g1 ## g2' ->

  (* σ is a finite map from the inputs of g2 to the inputs of g2' *)
  (forall id, id ∈ inputs g2 -> (σ id) ∈ inputs g2') ->
  (forall id, id ∉ inputs g2 -> (σ id) = id) ->
  
  (* The substituted subgraph is equivalent to the original one *)
  (forall from to, to ∉ nTO -> 
    ⟦g2⟧ (from,to) ≈ ⟦g2'⟧ (from, SIG to)) ->

  (* Then, the graphs are equivalent from any valid initial start *)
  forall from to, to ∉ nTO ->
    ⟦g2 ∪ g1⟧ (from,to) ≈ ⟦g2' ∪ ocfg_term_rename σ g1⟧ (from, σ to).
      \end{lstlisting}
    \label{fig:ocfg_equiv}
  \end{figure}
\end{frame}

\subsection*{BlockFusion}

% \begin{frame}[fragile]{Block Fusion}
%   \begin{figure}
%     \begin{lstlisting}[style=customcoq,basicstyle=\small\ttfamily]
% Definition BlockFusion_f {S}: (block_id * blk * (block_id * blk * S) -> bool) :=
%   fun '(_, A, (idB, B, _)) => is_seq A idB && no_phi B.

% Definition BlockFusion {S} (P: Pattern S) := When (Block (Head P)) BlockFusion_f
%   \end{lstlisting}
%     \caption{The \ilc{BlockFusion} pattern}
%     \label{fig:bfusion}
%   \end{figure}
% \end{frame}

\begin{frame}[fragile]
  \begin{figure}
    \begin{lstlisting}[style=customcoq,basicstyle=\small\ttfamily]
Definition σfusion idA idB := fun (id: block_id) => if decide (id=idA) then idB else id.

Definition fusion (σ: block_id -> block_id) (A B: blk): blk := {|
  blk_phis       := A.(blk_phis);
  blk_code       := A.(blk_code) ++ B.(blk_code);
  blk_term       := term_rename σ B.(blk_term)
|}.
    \end{lstlisting}
    \caption{The \ilc{fusion} function}
    \label{fig:fusion}
  \end{figure}
\end{frame}

\begin{frame}[fragile]
  \begin{figure}
    \begin{lstlisting}[style=customcoq,basicstyle=\small\ttfamily]
Lemma fusion_correct {S} G idA A idB B P (X:S):
  let σ := σfusion idA idB in
  (idA, A, (idB, B, X)) ∈ (MatchAll (BlockFusion P) G) ->
  forall f to : block_id, to ∉ {[idB]} ->
    ⟦ {[idA := A; idB := B]} ⟧ (f, to) ≈ ⟦ {[idB := fusion σ idA A B]} ⟧ (f, σ to).

Theorem Denotation_BlockFusion_correct {S} G idA A idB B f to P (X:S):
  let σ := σfusion idA idB in
  let G0 := delete idB (delete idA G) in
  to <> idB ->
  (idA, A, (idB, B, X)) ∈ (MatchAll (BlockFusion P) G) ->
  ⟦ G ⟧ (f, to) ≈ ⟦ <[idB:=fusion σ idA A B]> (ocfg_term_rename σ G0) ⟧ (f, σ to).
        \end{lstlisting}
    \caption{the correctness theorems of Block Fusion}
    \label{fig:fusion_proof}
  \end{figure}
\end{frame}

\section{Conclusion}

\begin{frame}{Conclusion}
  \begin{itemize}
    \item DSL of Patterns for specifying structural properties on subgraphs.
    \item Matcher function to identify subgraphs that match these properties.
    \item Generic substitution theorem to prove optimizations.
  \end{itemize}
\end{frame}

\begin{frame}{Thank you for listening.}
  Any questions?
\end{frame}

\end{document}