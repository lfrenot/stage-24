\documentclass[11pt]{article}

\usepackage[style=numeric, natbib]{biblatex}
\usepackage[margin=20ex]{geometry}
\usepackage{amsmath}
\usepackage{array}
\usepackage{booktabs}
\usepackage{caption}
\usepackage{fancyvrb}
\usepackage{graphicx}
\usepackage{hypdoc}
\usepackage{libertine}
\usepackage{longtable}
\usepackage{newtxmath}
\usepackage{tabularx}
\usepackage{zi4}
\usepackage{graphicx}
\usepackage{colortbl}
\usepackage{microtype}
\usepackage{balance}
\usepackage{mathtools}
\usepackage{float}
\usepackage{color}
\usepackage{listings,lstlangcoq}
\usepackage[dvipsnames]{xcolor}
\usepackage[all]{xy}
\usepackage{xspace}

\lstdefinestyle{customcoq}{
  columns=flexible,
  mathescape=true,
  belowcaptionskip=1\baselineskip,
  breaklines=true,
  xleftmargin=\parindent,
  language=Coq,
  morekeywords={Variant, fun, Arguments, Type, cofix},
  % morekeywords={SOCKAPI,ITREE,data_at,data_at_},
  emph={%
    SOCKAPI,ITree,data_at,data_at_
  },
  emphstyle={\bfseries\color{green!40!red!80}},
  showstringspaces=false,
  basicstyle=\small\ttfamily,
  keywordstyle=\bfseries\color{green!20!black},
  commentstyle=\itshape\color{red!40!black},
  identifierstyle=\color{violet!50!black},
  stringstyle=\color{orange},
  escapeinside={<@}{@>}
}
\newcommand{\inlinecoq}[1]{\mbox{\lstinline[style=customcoq,columns=fixed,basewidth=.48em]{#1}}}
\newcommand{\ilc}[1]{\inlinecoq{#1}}

\addbibresource{references.bib}

\newcommand{\leon}[1]{\textcolor{blue}{#1}}
\newcommand{\yz}[1]{\textcolor{ForestGreen}{#1}}
\newcommand{\yzt}[1]{\textcolor{ForestGreen!50}{#1}}
\newcommand{\cut}[1]{\textcolor{Gray!40}{#1}}

\newtheorem{theorem}{Theorem}
\newtheorem{definition}{Definition}
\newtheorem{example}{Example}

\newcommand{\ocfg}{OCFG\xspace}
\newcommand{\pat}{\texttt{Pat}\xspace}

\hypersetup{colorlinks=true, linkcolor=black}

\begin{document}

\title{CFG Patterns: A new tool to formally verify optimisations in Vellvm}

\author{Leon Frenot\\ Supervised by Yannick Zakowski \& Gabriel Radanne}

\date{February 5th, 2024 - July 5th, 2024}

\begin{titlepage}
  \centering
  {\textsc{École normale supérieure de Lyon} \par}
  \vspace{1cm}
  {\Large \textsc{Internship Report}\par}
  \vspace{1.5cm}
  {\huge\bfseries CFG Patterns: A new tool to formally verify optimisations in Vellvm\par}
  \vspace{2cm}
  {\Large\itshape Leon Frenot\par}
  \vfill
  supervised by\par
  Yannick~Zakowski~\&~Gabriel~Radanne\par
  at ENS Lyon
  \vfill

  % Bottom of the page
  {\large February 5th, 2024 - July 5th, 2024\par}
\end{titlepage}

\tableofcontents
\newpage

\hypersetup{colorlinks=true, linkcolor=red}

\begin{abstract}
  \leon{Abstract}
\end{abstract}

\section{Introduction}
\label{sec:intro}

\indent
For my M2 year at ENS de Lyon, I completed a 20 weeks internship in the LIP Computer Science laboratory in ENS Lyon. This internship was supervised by Yannick Zakowski and Gabriel Radanne in the Compilation and Analysis, Software and Hardware Laboratory of the LIP. The goal of this internship was to design and implement a pattern language over control flow graphs to provide a framework for formal proofs on optimizations.

Compilation

Compilation Vérifiée, i.e., préservation de la sémantique

Dans un assistant de preuve, Coq en l'occurence.

Des compilateurs vérifiées, y en a qui exisent. en particulier vellvm, donc laius llvm ir.

fondements sémantiques de vellvm === interpréteurs monadiques
raisonnement == bisimilarity

expérience avec vellvm === helix === un gros bazar avec les noms, et la génération de noms frais, etc\dots

Appelle à des abstractions pour programmer et raisonner sur des passes de génération et de transformation de graphes.

Expériences précédentes se sont penchés sur la conception de combinateurs pour construire des graphes, ce qui est pertinent pour un front end.

Ici, on s'intéresse à un problème complémentaire : lors de l'écriture de passes d'optimizations, il nous faut identifier un sous-graphe satisfaisant des propriétés structurelles particulières, y opérer une transformation, et réintégrer le sous-graphe transformé dans son contexte.

Contributions.

Compilers are extremely complex objects, gcc's source code is made of about 15 million lines of code. Because of that complexity, it can be hard to assert that compilers follow theor specification. For instance, code compiled for aeronautics is usually done without optimisations to reduce the likelyhood of unindented behaviors.

One possible solution to this dilema is !emph verified compilation. That is, compilers that are proven to preserve the semantics of the source code. Proving this implies defining formal definitions for the semantics of the source langage, the target langage and the proof of the preservation of behavior between them. Verified compilers usually rely on proof assistants like Coq, which allow writing executable code, i.e. the compiler, and formal proofs on that code.

Verified compilers have been a research subject for some time now, notbaly since the CompCert project, which established a verified compiler for the ISO C 99 langage. It is comercialised by !, for example for nuclear and aeronautics industries. Our work takes place in the context of the Vellvm project, which aims at formalizing the semantics of LLVM IR, an intermedite representation used as a front§end and back§end for many langages and architectures. LLVM IR is based on a Control§Flow Graph model, with named blocs and abstract registers.

Vellvm has had some applications, notably to prove some IR to IR optimizations, as well as to establish a verified front§end from Helix to LLVM IR. One key challenge that was identified from establishing that front§end was fresh identifier generation. This is a reoccuring issue with foraml proofs on named systems.

This observation calls for better abstraction for reasoning on graphs. Previous experiments aimed at designing combinators to ease the construction of graphs capturing high§lvel control flow abstractions, a solution suitable to write front§ends.

Our work focused on a orthogonal notion. When writing optimization passes, there is often a need to identify a subgraph with specific structural properties, apply a transformation on it, and link the transformed subgraph back with the context.

To adress this problematic, I designed a Domain Specific Langage (DSL) of patterns over graphs, ¡pat. It is equipped with a matcher which, given a graph, computes the set of decompositions that matches a given pattern. This matcher is proven to follow a specification capturing at a higher level the structural properties expressed by ¡pat.

Following this, I worte an implementation of the Block Fusion optimization based on ¡pat, and proved it preserves the semantics of the graph. To establish this proof, I have on the one hand proved a generic substitution theorem, and on the other hand leveraged the sepcification of a pattern to show the structural properties of a captured subgraph matches the requirements for the optimization. 

In the remaining of this report, Sectionë¡ref itnroduces the necessary prerequisites on Iteraction Trees and Vellvm. Section ref describes the ¡pat DSL and its semantics. And finally, Section ref convers the Block Fusion case study.

\paragraph{State of the Art}

\leon{Compilation certifiée}

\leon{transformation de code}

\leon{patterns et graphs}

\paragraph{The Contribution of This Work}

\begin{itemize}
  \item \leon{Design d'un DSL de patterns: \pat + Implémentation naive d'un matcher}
  \item \leon{Preuve d'un théorème central pour prouver des optims (sur un CFG)}
  \item \leon{Utiliser ce langage pour block fusion + preuve de correction}
\end{itemize}
\paragraph{Premier exemple: CCstP}

\section{Background}
\label{sec:background}

\subsection{Interaction Trees}

Interaction Trees (ITrees) are a coinductive structure designed to represent the dynamic behaviors of a computation. The goal of ITrees is to model recursive and effectful programs, including divergent computations.

Figure~\ref{fig:itree} show the Coq definition of ITrees. An instance of the type features an \emph{event} type \ilc{E: Type -> Type}, and a return type \ilc{R: Type}. The definition uses three constructors: \ilc{Ret} corresponds to halting and returning a value of type \ilc{R}; \ilc{Tau} corresponds to a \emph{silent step}, i.e. an internal computation, followed by computation \ilc{t}; and \ilc{Vis} corresponds to a \emph{visible event}, it describes an external computation \ilc{e} which returns a value of type \ilc{A}, and a continuation \ilc{k} which depends on that return value.

\begin{figure}
  \begin{lstlisting}[style=customcoq,basicstyle=\small\ttfamily]
CoInductive itree (E : Type -> Type) (R : Type) : Type :=
  | Ret (r : R)
  | Tau (t : itree E R)
  | Vis {A : Type} (e : E A) (k : A -> itree E R).
  \end{lstlisting}
  \caption{The \ilc{itree} datatype}
  \label{fig:itree}
\end{figure}

To reason over ITrees, we have multiple notions of \emph{bisimulation}. The most relevant one is \emph{weak bisimulation}, noted \ilc{t1 ≈ t2}. We say that \ilc{t1 ≈ t2} if they return the same value, and have the same visible events. This relation is an equivalence ``up-to-Tau'' in the sense that we have \ilc{Tau t ≈ t} and \ilc{t ≈ Tau t}. This equivalence can be refined up to a relationship $R : A \rightarrow B \rightarrow Prop$: we then have $\ilc{Ret}\; a \approx_R \ilc{Ret}\; b \iff R\; a\; b$. 

Effects can easily be added or removed from the semantics of an ITree. The \ilc{Vis} constructor represents \emph{uninterpreted events}. By defining an \emph{event handler}, semantics are assigned to these events. Interpreting an ITree then consists of folding that handler over the ITree. This allows the semantics of ITrees to be \emph{modular}.

Furthermore, the semantics of ITrees are also \emph{compositional} with the use of \emph{combinators}. For example, \ilc{bind: itree E A -> (A -> itree E B) -> itree E B} allows composing ITrees (with the use of continuations).
Other combinators include \ilc{iter: (A -> itree E (A+B)) -> (A -> itree E B)} to encode the iterations and hide the co-induction (by hiding each body step in a Tau), and \ilc{mrec} for mutual-recursive combinators. 

Unlike similar projects, which rely on \emph{operational semantics} and simulation diagrams, ITrees rely on \emph{denotational semantics}. It is based on equations that can be used to prove bisimulation. These equations allow the user to reason without using coinductive reasoning or the definition of the weak bisimulation. The compositionality of the semantics also allow simpler reasoning than operational semantics, since program counter and similar notions are lifted away.

\subsection{Vellvm}

\leon{TODO: figures (vir syntax and intrep stack), emphs}

The goal of the Vellvm project is to \emph{formally define} the semantic of the LLVM IR and construct verified
components for that formalization.

LLVM is a compiler infrastructure designed around a language-independent \emph{intermediate representation} (IR). It is used to develop frontends for programming languages and backends for instruction set architectures.

The LLVM IR is an instruction set similar to low-level assembly languages like RISC, but features high-level informations. This duality allows it to represent any program while still permitting analysis and optimizations. It is based on control flow-graphs, with named labels and registers, and guarantees Single Static Assignment (SSA) form, which is key to many static analyses and optimizations. The LLVM IR is statically typed, and features integer-pointer casts.

Optimizations and analysis on the LLVM IR are done through successive \emph{analysis and transformation passes}.

Vellvm introduces Verified LLVM IR (VIR), a \emph{realistic subset} of the LLVM IR. Figure~\ref{vir} shows a subset of VIR's syntax. VIR's semantics are defined with ITrees: each element of VIR's syntax is represented by a corresponding ITree. Each effect (except control flow) is captured by a \ilc{Vis} event, which can be interpreted later. This semantic includes many non-trivial features of LLVM IR, including pointers, LLVM's phi-nodes and undefined behaviors.

Since the semantics of a block or set of blocks can be defined without relying on a ``complete'' CFG, it is possible to use ``open control-flow graphs'' (\ocfg), which is simply a set of blocks without a defined entry point.

To interpret the semantics of the different effects of its syntax, Vellvm uses a stack of interpreters. It gradually introduces external elements to the semantic (intrinsics, global and local environments, \ldots). Figure~\ref{interp} show that stack of interpretation. The final levels split between a \emph{propositional} model, which interprets the non-determinism of LLVM IR's undefined behaviors, and an \emph{executable} model, which implements one of these behaviors.

\section{The pattern language}
\label{sec:lang}

% \yz{De façon générale, commencer par des bullet points ou des séries de paragraphe est excellent, mais c'est important de lui donner corps en rédigeant dans un second temps un texte cohésif. Je tente une proposition rapide par exemple pour ce chapeau pour illustrer.}

\yz{Remarque générale : il faut utiliser beaucoup beaucoup de macros quand on écrit du TeX. Par exemple OCFG va apparaître beaucoup, et on peut hésiter sur la façon de le typeset/écrire : macro }

\yz{Il peut être pratique d'avoir un nom pour ton langage pour pouvoir y référer.}

\yzt{We now turn our attention to the central piece of our contribution: the design of \pat{}, a DSL of patterns for writing and proving correct program transformations. This DSL is composed of two core components. First, an indexed datatype provides a syntax for the user to specify how they wish to decompose an input \ocfg. Second, a \emph{matcher} provides a semantics to the language, specifying the valid decompositions associated to each pattern. Finally, we illustrate on an example the definition and semantic characterization of a pattern extracting the heads of a graph, written in our DSL.}

In this section we will focus on our main contribution: the design of \pat{}, a DSL of patterns for writing and proving correct program transformations. This DSL is composed of two core components. The first is a datatype, \ilc{Pattern}, which provides a syntax for the user to specify how they wish to decompose an input \ocfg. The second is a \emph{matcher} function, which provides semantics to apply the language to an \ocfg{}, specifying the valid decompositions associated to each pattern. Finally, we illustrate on an example the definition and semantic characterization of a pattern extracting the heads of a graph, written in our DSL\@.

\cut{
In this section we will:\begin{itemize}
  \item Define a Domain Specific Language that can capture optimizable subgraphs in an \ocfg\@.
  \item Introduce a matcher on this language and the corresponding semantics of each constructor.
  \item Present the Coq implementation of the language, matcher and semantics.
\end{itemize}
}

\subsection{\pat: a DSL for pattern matching on graphs}

\yzt{At a high level, we look for a language allowing the user to characterize and reason about optimizable subgraphs in an \ocfg. 
To this end, we introduce \pat, a general, very expressive DSL for pattern matching on graphs. The specific patterns we are interested in from the perspective of compilation will then be expressed in \pat.}

\begin{figure}
  \begin{lstlisting}[style=customcoq,basicstyle=\small\ttfamily]
    Inductive Pattern : Type -> Type :=
    | Graph: Pattern ocfg
    | When: forall  {S}, Pattern S -> (S -> bool) -> Pattern S
    | Map: forall  {S} {T}, Pattern S -> (S -> T) -> Pattern T
    | Focus: forall  {S}, Pattern S -> Pattern (ocfg * S)
    | Block: forall  {S}, Pattern S -> Pattern (bid * blk * S)
    | Head: forall  {S}, Pattern S -> Pattern (bid * blk * S)
    | Branch: forall  {S}, Pattern S -> Pattern (bid * blk * S)
  \end{lstlisting}
  \caption{The \ilc{Pattern} datatype}
  \label{fig:pat}
\end{figure}

Figure~\ref{fig:pat} introduces \pat{}'s syntax, defined as an inductive datatype \ilc{Pattern}.
Because the purpose of a pattern is to decompose a graph into a certain structure, the \ilc{Pattern} datatype reflects this intention by taking as argument a type, which represents the return type of the pattern.\footnote{Such families of types are common in dependently typed languages, and are referred to as Generalized Algebraic Data Types in languages such as OCaml or Haskell.}
This typing information is leveraged in the definition of the matcher, introduced in Section~\ref{sec:matcher}: a pattern of type \ilc{Pattern S} will be matched against elements of type \ilc{S}.

The \ilc{Pattern} type is build of seven constructors. The only base case is the \ilc{Graph} constructor which trivially match any graph and does not perform any decomposition. On more traditional paper presentation, it corresponds to a single hole $\square$.

The six other constructors recursively decompose the graph, typically enriching the return type of the pattern in doing so. 
The \ilc{When} constructor acts as a filter:
given a pattern of return type \ilc{S}, it builds a pattern with the same return type, but takes as argument a filtering function \ilc{S -> bool} used to restrict the set of matching graphs to those satisfying the condition.
The \ilc{Map} constructor simply hardcodes functoriality into the datatype, allowing for post-processing the output of a pattern by a pure function. That is: given a return type \ilc{S} and a function \ilc{S -> T}, it builds a pattern of return type \ilc{T}.
The \ilc{Focus} constructor, given a pattern \ilc{P}, builds a pattern that tries to match \ilc{P} on any subset of the graph given as argument.
The \ilc{Block} constructor extracts any block from the graph and matches the pattern given as argument against the rest of the graph.
The \ilc{Head} constructor extracts a block without predecessors from the graph and matches the pattern given as argument against the rest of the graph.
The \ilc{Branch} constructor extracts a block whose terminator is a branch from the graph and matches the pattern given as argument against the rest of the graph.

This set of patterns is not minimal. Indeed, \ilc{fun p => Block p} could be written as\\\ilc{fun p => When (Focus p) (fun '(l, _) => size l = 1)}, and \ilc{fun p => Branch p} could be written as \ilc{fun p => When (Block p) (fun '(id, b, _) => term_is_branch b)}. Finding a minimal set was not a goal of the DSL as the main concern was ease of use, both for the user, for designing a matcher, and for establishing the meta-theory. 

\ilc{Map} and the functoriality it brings feel natural to add and are easy to implement and prove. But we have no immediate use for it at this time. An argument against \ilc{Map} could be made, as every other constructor allows working backwards to rebuild the graph by recombining the extracted part (as long as it matches with something).

\ilc{Map} also makes \ilc{Head} a minimal constructor. Indeed, while \ilc{Head Graph} can be written as \ilc{When (Block Graph) (fun '(id, b, g) => is_head id g)}, there is no function f such that\\\ilc{Head Map Graph (fun _ => ())} can be written as \ilc{When (Block Map (fun _ => ())) f}, since we loose the information on the rest of the graph.

While a user will want to work using closed instances of pattern, that is objects of type \ilc{Pattern S}, our proofs work on open instances of pattern, that is objects of type\\\ilc{Pattern S -> Pattern (f S)}, to allow compositionality. We will call these objects \emph{opened patterns} \leon{ou open patterns?}. Note that constructors other than \ilc{Graph} are opened patterns.

\paragraph*{An example of pattern: block fusion.}
We illustrate the use of \pat{} to implement\footnote{Or rather, to \emph{specify} such an optimization. We discuss briefly executability in conclusion.} a bloc fusion optimization.
Block Fusion typically consists of finding two blocks (A and B) such that A is the only predecessors of B, B is the only successor of A, and B is not the graph's entry point. Since an execution of A is always followed by B, and an execution of B is always preceded by A, we can replace them by a single block that follows the execution of both A and B.

The applicable subgraphs are specified with the pattern\\\ilc{pfusion := When (Block (Head Graph)) BlockFusion\_f}.
The pattern starts by the \ilc{Block} to match any block \ilc{A} Then, \ilc{Head} matches a block \ilc{B} that has no predecessors (except possibly \ilc{A} as it is not in scope of the pattern anymore), and finally \ilc{When _ BlockFusion\_f} asserts that \ilc{B} is the only successor of \ilc{A}.

\begin{figure}
  \xymatrix{
    &&&&\\
    &*+[F]\txt{\ilc{A}\\\ilc{Block}: any block}\ar[dd]&&\;\ar@{-->}`u[ul] `[ll] [ll]&\\
    *+[F.]\txt{\ilc{When _ BlockFusion_f}\\ asserts that \ilc{B} is\\the only successor of \ilc{A}}\ar@{..>} [r]&&&\txt{Graph}\\
    &*+[F]\txt{\ilc{B}\\\ilc{Head}: no predecessors\\once \ilc{A} is removed}\ar@{-->}`d[dr] `[rr] [rr]&&&\\ 
    &&&
    \save "2,3"."4,5"*[F--]\frm{} \restore
  }
  \caption{The \ilc{pfusion} pattern}
  \label{fig:fusion}
\end{figure}

Figure~\ref{fig:fusion} illustrates graphically the shape of the graph decompositions that match \ilc{pfusion}. The full lines (\ilc{A, B} and the arrow between them) represent the parts fixed by \ilc{pfusion}. The dashed lines refer to parts that are allowed, but not necessarily them: other blocks in the graphs with arrows coming towards \ilc{A} or from \ilc{B} (but not in the middle of them). The dotted lines 
show the \ilc{When} constructor.
% \yz{Refer to and explain the figure! "Figure~2 illustrate graphically the shape of the graph decompositions that match \ilc{pfusion} ..." Use the names suggested above, and explain the meaning of the different arrows.}

\subsection{Matcher functions}
\label{sec:matcher}

We now turn to the question of defining the semantics of our patterns, via a \emph{matcher function}. 
A matcher function takes a pattern \ilc{p} and an \ocfg{} \ilc{G} as arguments, and returns a set of decompositions of \ilc{G} that match \ilc{p}.
We have mainly focused on \emph{specifying} patterns, by implementing a matcher that returns \emph{all} valid decompositions. 
Although rather meant as a specification than as a realistic implementation, the \ilc{MatchAll} function, depicted in Figure~\ref{fig:matchall}, is nonetheless executable. This not only allows for testing, but also drastically reduces the distance left to bridge to have an efficient implementation that is provably sound. To do so, the matcher lives in the list monad: it computes recursively the decomposition according to the sub-pattern, and flat maps (Definition provided on Figure~\ref{fig:flatmap}) a function for each constructor extending the decomposition according to the head constructor.

\begin{figure}
  \label{fig:match}
  \begin{lstlisting}[style=customcoq,basicstyle=\small\ttfamily]
Fixpoint MatchAll {S} (P: Pattern S) (g: ocfg) : list S :=
  match P with
    | Graph => [g]
    | When p f => filter (fun x => f x = true) (MatchAll p g) 
    | Map p f => map f (MatchAll p g)
    | Focus p => flat_map_r (MatchAll p) (focus g)
    | Block p => flat_map_r (MatchAll p) (blocks g)
    | Head p => flat_map_r (MatchAll p) (heads g)
    | Branch p => flat_map_r (MatchAll p) (branches g)
  end.
  \end{lstlisting}
  \caption{The \ilc{MatchAll} function}
  \label{fig:matchall}
\end{figure}

\begin{figure}
  \begin{lstlisting}[style=customcoq,basicstyle=\small\ttfamily]
Definition flat_map_r {A B C} (f : B -> list C) (l : list (A*B)) : list (A*C) :=
  match l with
    | [] => []
    | (a, b)::q => (map (fun c => (a, c)) (f b))++flat_map_r f q
end.
  \end{lstlisting}
  \caption{The \ilc{flat_map_r} function}
  \label{fig:flatmap}
\end{figure}
\begin{figure}
  \begin{lstlisting}[style=customcoq,basicstyle=\small\ttfamily]
Definition focus_aux id b acc : list ocfg := acc ++ (map (insert id b) acc).

Definition focus (G: ocfg) := map (fun G' => (G', G ∖ G')) (map_fold focus_aux [∅] G).
  \end{lstlisting}
  
  \begin{lstlisting}[style=customcoq,basicstyle=\small\ttfamily]
Definition blocks_aux (G: ocfg) : (bid*blk) -> (bid*blk*ocfg) :=
  fun '(id, b) => (id, b, delete id G).

Definition blocks (G: ocfg): list (bid*blk*ocfg) :=
  map (blocks_aux G) (map_to_list G).
  \end{lstlisting}
  
  \begin{lstlisting}[style=customcoq,basicstyle=\small\ttfamily]
Definition heads_aux (G: ocfg) id b acc : list (bid*blk*ocfg) :=
  if is_empty (predecessors id G)
  then (id, b, delete id G)::acc
  else acc.

Definition heads (G: ocfg): list (bid*blk*ocfg) := map_fold (heads_aux G) [] G.
  \end{lstlisting}
  
  \begin{lstlisting}[style=customcoq,basicstyle=\small\ttfamily]
Definition branches_aux (G: ocfg) id b acc : list (bid*blk*ocfg) :=
  match b.(blk_term) with
    | TERM_Br _ l r => (id, b, (delete id G))::acc
    | _ => acc
  end.

Definition branches (G: ocfg): list (bid*blk*ocfg) := map_fold (branches_aux G) [] G.
  \end{lstlisting}
  \caption{\ilc{MatchAll}'s subfunctions}
  \label{fig:match_sub_funs}
\end{figure}

For the extracting constructors, \ilc{Focus, Block, Head} and \ilc{Branch}, \ilc{MatchAll} also relies on decomposition functions, whose definitions are given in Figure~\ref{fig:match_sub_funs}. These functions each rely on an auxiliary function which is folded over the graph to find each matching element.

\subsection{Specification}

\yzt{We have claimed that \ilc{MatchAll} \emph{returns all valid decompositions}. We capture this statement by providing specifications for each constructor of \pat, and proving they are exactly captures by \ilc{MatchAll}.}

A specification for a opened pattern \ilc{PatF} of type \ilc{Pattern S -> Pattern (f S)} is a function \ilc{R_PatF} of type \ilc{ocfg -> Pattern S -> f S -> Prop}. We say that \ilc{MatchAll} is correct for that specification if we have:\\\ilc{forall (G: ocfg) (pat: Pattern S) (X:f S), X IN MatchAll (PatF pat) G IFF R_PatF G pat X}.

Similarly, a specification for a closed pattern \ilc{pat: Pattern S} is \ilc{R_pat: ocfg -> S -> Prop}. \ilc{MatchAll} is correct for it if: \ilc{forall (G:ocfg) (X:S), X IN MatchAll pat G IFF R_pat G X}. 

The correction for \ilc{Graph} is given by the following theorem:\\\ilc{forall G G', G' ∈ (MatchAll Graph G) IFF G' = G.} Its proof is immediate by definition of \ilc{MatchAll Graph}.

The correction for \ilc{fun P f -> When P f} is given by the following theorem:\\\ilc{forall (P: Pattern S) f X G, X ∈ (MatchAll (When P f) G) IFF f X = true \/\\ X ∈ (MatchAll P G).} We can also give a theorem for the correctness of \ilc{Map}. Proving their correctness is then immediate using builtin lemmas on \ilc{filter} and \ilc{map}.

The same proof mechanism is used for \ilc{Focus}, \ilc{Block}, \ilc{Head} and \ilc{Branch}. We will now detail it for \ilc{Head}.

% \ilc{MatchAll} relies on the \ilc{heads} function to match the \ilc{Head} constructor. The goal of that function is to find all the "heads", i.e. blocks without predecessors, in an OCFG. To do that, it folds a \ilc{heads_aux} function over the map. That function calls the \ilc{predecessors} function on each block, and appends the result to the return list if the block doesn't have predecessors.

Figure~\ref{fig:sem_head_def} gives a semantic definition for \ilc{Head}. It is composed of two parts: \ilc{heads_aux_sem} which gives semantics for the auxiliary function, which is used to prove its correctness, and \ilc{heads_sem} which gives the actual semantics for \ilc{Head} and is used 

\begin{figure}
  \begin{lstlisting}[style=customcoq,basicstyle=\small\ttfamily]
Record heads_aux_sem (G0 G G': ocfg) id b := {
  EQ: G' = delete id G0;
  IN: G !! id = Some b;
  PRED: predecessors id G0 = ∅
}.

Definition heads_sem (G G':ocfg) (id:bid) b := heads_aux_sem G G G' id b.
  \end{lstlisting}
  \caption{The semantic definition for \ilc{Head}}
  \label{fig:sem_head_def}
\end{figure}

Finally, as show in Figure~\ref{fig:head_cor} we can prove the semantics for the auxiliary function, the \ilc{heads} function and \ilc{MatchAll Head}. We first prove \ilc{heads_aux_correct} by induction on \ilc{G}, then \ilc{heads_correct} and \ilc{Pattern_Head_correct} follow immediately.

\begin{figure}
  \begin{lstlisting}[style=customcoq,basicstyle=\small\ttfamily]
Lemma heads_aux_correct:
  forall G G' G0 id b,
  (id, b, G') ∈ (map_fold (heads_aux G0) [] G) IFF heads_aux_sem G0 G G' id b.

Lemma heads_correct:
  forall G G' id b,
  (id, b, G') ∈ (heads G) IFF heads_sem G G' id b.

Theorem Pattern_Head_correct {S}:
  forall (G: ocfg) (P: Pattern S) id b X,
  (id, b, X) ∈ (MatchAll (Head P) G) IFF
  exists G', heads_sem G G' id b /\ X ∈ (MatchAll P G').
  \end{lstlisting}
  \caption{The semantic proof of \ilc{MatchAll Head}}
  \label{fig:head_cor}
\end{figure}

\section{Formal verification of an optimization pass: the case of block fusion}
\label{sec:deno}

\yzt{We have introduced in Section~\ref{sec:lang} a pattern for detecting in a graph pairs of blocks that may be fused. In this section, we turn our attention to using this pattern for defining the block fusion optimization itself, and consider its proof of correctness. Rather than establishing an ad-hoc proof, we first establish a more general theorem for justifying a broad class of program transformations, and apply it to the case of block fusion.}
\cut{In this section we will informally define an optimization class, show a theorem for proving the correctness of optimizations of that class, and apply this theorem to an implementation of Block Fusion.}

Recall from Section~\ref{sec:background} that we write $t \approx_R u$ to express $t$ and $u$ are weakly bisimilar itrees, whose leafs are related by the relation $R$, and $\approx$ when $R=eq$. On the Vellvm side, we write $\llbracket G \rrbracket$ for the denotation of a (possibly open) LLVM IR control flow graph into a function from pairs of block identifiers to itrees. Note that we consider here the first layer of the denotation: effects remains uninterpreted.
In the remaining of this section, we overload the equivalence notations between graphs: $G_1 \approx_R G_2$ should be understood as $\llbracket G_1\rrbracket \approx_R \llbracket G_2\rrbracket$.

\subsection{On subgraph substitution under context}

The purpose of a pattern is to identify a subgraph of the adequate shape, and optimize it: we hence focus here on optimizations that only modify a subcomponent of the graph---as opposed to graph-wide transformations that may modify everything, such as constant propagation. We hence need to consider two problems: capturing the notion of equivalence of graph we will consider between the patterned subgraph and its replacement, and justifying that this equivalence respects the context.

\yz{À nouveau, une figure flottante n'est pas faite pour être considérée comme partie linéaire du texte. Il faut y faire référence dans le texte, et l'expliquer!!}

\yz{Vraiment, ne force pas des retours à la ligne, ce n'est pas comme cela que LaTeX a été pensé.}

% \begin{figure}
%   \begin{lstlisting}[style=customcoq,basicstyle=\small\ttfamily]
% Theorem (g1 g2 g2' : ocfg):
%   forall from to, ⟦g2⟧bs (from,to) ≈ ⟦g2'⟧bs (from, to) ->
%   forall from to, ⟦g2 ∪ g1⟧bs (from,to) ≈ ⟦g2' ∪ g1⟧bs (from, to).
%   \end{lstlisting}
% \end{figure}

\yzt{Weak bisimilarity of interaction trees are proven to respect all relevant contexts in Vellvm. One could therefore simply consider the substitution of equivalent subgraphs $G_2$ and $G_2'$ in a larger context $G$, where graphs are equivalent if their denotations are bisimilar. Represented graphically:}
\xymatrix{
  &&&&&&&&\\
  G_2&\approx&G_2'&\implies&*+[F]{G_2}\ar@{-->}`d[dr] `[r] [r]&*+[F]{G}\ar@{-->}`u[ul] `[l] [l]&\approx&*+[F]{G_2'}\ar@{-->}`d[dr] `[r] [r]&*+[F]{G}\ar@{-->}`u[ul] `[l] [l]\\
  &&&&&&&&
}

\yzt{However, this convenient theorem is unusable for our purpose. Let us consider the case of block fusion: we replace two blocs $b_1$ and $b_2$ by a block $b$.
If we chose as block identifier (bid) for $b$ the same as for $b_2$, then we must reflect on the context the need to enter the subgraph from a different entry point.
In contrast, if we chose as bid for $b$ the same as for $b_1$, then we must reflect on the context that we exit the subgraph from a different entry point---as block provenance determines the semantics of phi-nodes.}

\yzt{Hence, we must take some renaming into account, whether at the subgraph's input interface, or its output interface. Both are viable, but we favor the former. First because performing and reasoning about the renaming of terminators (to account for renamed inputs) is marginally simpler than of phi-nodes (to account for renamed outputs). Second because the resulting obligation for the two equivalent subgraphs becomes simpler to spell out: we need to start the denotations from renamed bids, but can therefore establish an exact weak bisimulation.}

We start by defining the lifting of a renaming function \ilc{SIG} on bids to one on graphs \ilc{ocfg_term_rename SIG} by applying it to each of its blocks's terminators.
We define a function \ilc{ocfg_term_rename} which, given a function over ids \ilc{SIG} and a graph \ilc{g}, returns \ilc{g} with \ilc{SIG} applied to each id in its blocks' terminators.
We use this renaming function to establish a first contextual lemma: if two graphs $g_2$ and $g_2'$ are equivalent when executed from \ilc{SIG} related entry points, then the same holds true when extending the graphs with an arbitrary context $g_1$, provided it is renamed on the other side. That is:
  \begin{lstlisting}[style=customcoq,basicstyle=\small\ttfamily]
forall (g1 g2 g2' : ocfg) (σ : bid -> bid):
  forall from to, ⟦g2⟧bs (from,to) ≈ ⟦g2'⟧bs (from, SIG to) ->
  forall from to, ⟦g2 ∪ g1⟧bs (from,to) ≈ ⟦g2' ∪ ocfg_term_rename σ g1⟧bs (from, SIG to).
  \end{lstlisting}

This second statement remains insufficient to show the soundness of an optimization such as block fusion. Indeed, if we try start executing from the fused block $b_2$, the semantics are obviously different, one executing the code from $b_1$ and not the other.

More generally, we need to constrain the valid entry blocks of the graph. 
To do so, we introduce two sets of bids \ilc{nTO} and \ilc{nFROM} to restrict the equations to valid input and origin bids by listing the invalid inputs and origin bids.
With this in hand, we can state the main contextual theorem we establish, displayed on Figure~\ref{fig:ocfg_equiv}.
The first series of hypotheses are well-formedness conditions expressing that the context ($g_1$) and the subgraphs ($g_2$ and $g_2'$) are disjoint,
capturing that the invalid sources ($nFROM$) and entries ($nTO$) belongs to the subgraphs,
and establishing necessary conditions for the coinductive arguments.
The second group of hypothesis simply characterizes the fact that our renaming is a finite renaming from the domain of $g_2$ to the domain of $g_2'$, despite its very lose typing as a total function.
Finally, the third hypothesis is the semantically interesting one: the two subgraphs are equivalent at valid inputs.
Given these, we conclude to the equivalence at valid inputs of the overall graphs.

The formal proof is quite technical to mechanize, but relatively intuitive. We sketch here its high level structure, for the interested reader. We proceed by coinduction, and are in one of two cases:
\begin{itemize}
\item if we start the computation in $g_1$, we compute the same block on each side. We hence match them thanks to an up-to bind principle, synchronize both computations after the next jump, and conclude by coinduction;
\item if we start the computation in $g_2$, we prove and use a meta-theorem for Vellvm establishing that the denotation of a graph is equivalent to the denotation of any of its subgraphs, followed by the denotation of the whole graph. We can hence see each side of the computation as starting by computing the denotation of respectively $g_2$ and $g_2'$: by up-to bind principle, we can match their prefixes using the hypothesis. One last hiccups need to be accounted for: in all generality, $g_2$ and $g_2'$ could be pure computations, and therefore contain no coinductive guard: we cannot conclude by coinduction immediately.
Instead, with additional meta-theory from the itrees, we enrich the intermediate postcondition to know that we are exiting out of $g_2/g_2'$, and therefore are back to the case (1) where we can play the game a second time and conclude.
\end{itemize}

% \begin{figure}
%   \begin{lstlisting}[style=customcoq,basicstyle=\small\ttfamily]
% Theorem (g1 g2 g2' : ocfg) (σ : bid -> bid) (nFROM nTO: gset bid):
%   (forall from to, to ∉ nTO -> from ∉ nFROM -> ⟦g2⟧bs (from,to) ≈ ⟦g2'⟧bs (from, SIG to)) ->
%   forall from to, to ∉ nTO -> from ∉ nFROM -> ⟦g2 ∪ g1⟧bs (from,to) ≈ ⟦g2' ∪ ocfg_term_rename σ g1⟧bs (from, SIG to).
%   \end{lstlisting}
% \end{figure}

% Finally, we need some conditions to make sure that:\begin{itemize}
%   \item the unions are well-formed,
%   \item \ilc{nFROM} and \ilc{nTO} are preserved during the (coinductive) proof,
%   \item \ilc{SIG} only changes ids from \ilc{g2} to \ilc{g2'}.
% \end{itemize}

% These conditions give us the following final theorem:

\begin{figure}
\begin{lstlisting}[style=customcoq,basicstyle=\small\ttfamily]
Theorem denote_ocfg_equiv (g1 g2 g2' : ocfg) (σ : bid -> bid) (nFROM nTO: gset bid) :
    (* Well formedness properties, where ## is disjointness of sets or map domains *)
    (inputs g2 ∩ inputs g2') ## nFROM -> 
    nFROM ⊆ inputs g2 ∪ inputs g2' ->
    inputs g2' ∖ inputs g2 ⊆ nTO -> 
    nTO ⊆ inputs g2 ∪ inputs g2' -> 
    nTO ## outputs g1 ->
    g1 ## g2 -> 
    ocfg_term_rename σ g1 ## g2' ->

    (* σ is a finite map from the inputs of g2 to the inputs of gs2' *)
    (forall id, id ∈ inputs g2 -> (σ id) ∈ inputs g2') ->
    (forall id, id ∉ nFROM -> (σ id) = id) ->

    (* The subtituted subgraph is equivalent to the original one *)
    (forall from to, to ∉ nTO -> from ∉ nFROM -> 
           ⟦g2⟧bs (from,to) ≈ ⟦g2'⟧bs (from, SIG to)) ->

    (* Then, the graphs are equivalent from any valid initial start *)
    forall from to,
    to ∉ nTO -> from ∉ nFROM ->
    ⟦g2 ∪ g1⟧bs (from,to) ≈ ⟦g2' ∪ ocfg_term_rename σ g1⟧bs (from, σ to).
  \end{lstlisting}
  \caption{The \ilc{denote_ocfg_equiv} theorem}
  \label{fig:ocfg_equiv}
\end{figure}

\subsection{Motivations for Block Fusion}

In this section, we will define the Block Fusion optimization, describe a corresponding OCFG pattern,
and outline the proof of correctness of the optimization using the pattern.

The Block Fusion optimization consists of picking two blocks $A$ and $B$,
such that $A$ is the only predecessor of $B$ and $B$ is the only successor of $A$,
and replacing them with a single block containing the code of $A$ and $B$.

This optimization is relevant for three main reasons:\begin{itemize}
  \item It is a commonly used optimization, for example to clear blocks created while building SSA form.
  \item It is an optimization that modifies the graph.
  \item It is simple to prove on paper that the optimization is correct.
\end{itemize}

In the previous section, we already gave a pattern for \ilc{BlockFusion}, we will use a slight variation, which allows further composing:\\\ilc{Definition BlockFusion {S} (P: Pattern S) := When (Block (Head P)) BlockFusion_f.}

% \subsection{The semantics of Block Fusion}

\ilc{BlockFusion_f} has two conditions:\begin{itemize}
  \item the terminator of the first block is an absolute jump to the second block,
  \item the second block does not have phi nodes.
\end{itemize}

The first condition is needed (instead of just checking the successors) because, if there is a conditional jump, evaluating the condition may lead to an error, and so to a difference in semantic after the fusion.\\
The second condition is needed because of the difference in evaluation between phi-nodes and assignment operations.

With this, we can create a \ilc{fusion} function for Block Fusion (\ilc{term_rename} applies \ilc{SIG} to each id in the terminator).

\begin{figure}
  \begin{lstlisting}[style=customcoq,basicstyle=\small\ttfamily]
Definition fusion (σ: bid -> bid) (idA : bid) (A B: blk): blk := {|
  blk_phis       := A.(blk_phis);
  blk_code       := A.(blk_code) ++ B.(blk_code);
  blk_term       := term_rename σ B.(blk_term);
  blk_comments   := fusion_comments A B
|}.
  \end{lstlisting}
  \caption{The \ilc{fusion} function}
\end{figure}

We also define \ilc{SIGfusion}, the renaming function for Block Fusion:

\begin{figure}
  \begin{lstlisting}[style=customcoq,basicstyle=\small\ttfamily]
Definition σfusion idA idB := fun (id: bid) => if decide (id=idA) then idB else id.
  \end{lstlisting}
\end{figure}

With these, we can prove first that \ilc{fusion} is correct, and then that the Block Fusion optimization is correct.

\begin{figure}
  \begin{lstlisting}[style=customcoq,basicstyle=\small\ttfamily]
Theorem Denotation_BlockFusion_correct {S} G idA A idB B f to P (X:S):
  let σ := σfusion idA idB in
  let G0 := delete idB (delete idA G) in
  to <> idB ->
  f <> idA ->
  (idA, A, (idB, B, X)) ∈ (MatchAll (BlockFusion P) G) ->
  ⟦ G ⟧bs (f, to) ≈ ⟦ <[idB:=fusion σ idA A B]> (ocfg_term_rename σ G0) ⟧bs (f, σ to).
      \end{lstlisting}
\end{figure}

\section{A voir: Approfondissements}
\label{sec:appr}

\subsection{Loop pattern}

\xymatrix{
  \ar[d]&&&\ar[d]\\
  *+[F]\txt{A}\ar@/^/[d]& & & *+[F]\txt{B}\ar@/^/[d]\\
  *+[F]\txt{B}\ar@/^/[u] \ar@/_/[dr]\ar[d]& & \Rightarrow & *+[F]\txt{A}\ar@/^/[u] \ar@/_/[dr]\ar[d]\\
  &*+[F]\txt{*}\ar@/_/[ul]& & & *+[F]\txt{*}\ar@/_/[ul]
}

\xymatrix{
  \ar[dd]&&&\ar[d]\\
  &&&*+[F]\txt{*}\ar[d]\\
  *+[F]\txt{A}\ar@/_/[dr]\ar[d]&&\Rightarrow&*+[F]\txt{A}\ar@/_/[dr]\ar[d]\\
  &*+[F]\txt{*}\ar@/_/[ul]&&&*+[F]\txt{*}\ar@/_/[ul]
}

\xymatrix{
  \ar[dd]&&&\ar[d]\\
  &&&*+[F]\txt{A}\ar@/_/[r]\ar[d]&*+[F]\txt{1}\ar@/_/[l]\\
  *+[F]\txt{A}\ar@/_/[dr]\ar[d]&*+[F]\txt{2}\ar@/_/[l]&\Rightarrow&*+[F]\txt{A'}\ar@/_/[r]\ar[d]&*+[F]\txt{2}\ar@/_/[l]\\
  &*+[F]\txt{1}\ar@/_/[u]&&&
}

\xymatrix{
  \ar[dd]&&&\ar[d]\\
  &&&*+[F]\txt{1}\ar[d]\\
  *+[F]\txt{A}\ar@/_/[dr]\ar[dd]&*+[F]\txt{2}\ar@/_/[l]&\Rightarrow&*+[F]\txt{A}\ar@/_/[dr]\ar[d]&*+[F]\txt{1}\ar@/_/[l]\\
  &*+[F]\txt{1}\ar@/_/[u]&&*+[F]\txt{2}\ar[d]&*+[F]\txt{2}\ar@/_/[u]\\
  &&&&&
}

\subsection{Other interpretation levels}

\subsection{Optim efficace}

\section*{Conclusion}
\label{sec:ccl}

\end{document}